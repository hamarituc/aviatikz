\documentclass[11pt,a4paper]{article}


\usepackage[T1]{fontenc}
\usepackage[utf8]{inputenc}
\usepackage{newpxtext}
\usepackage{newpxmath}
\usepackage{parskip}
\usepackage{a4wide}
\usepackage{float}
\usepackage{subcaption}
\usepackage{tikz}
\usepackage{aviatikz}
\usepackage{xfrac}
\usepackage{amsmath}
\usepackage{siunitx}
\usepackage{listings}
\usepackage{listingsutf8}

\floatplacement{figure}{htb}

\DeclareSIUnit[number-unit-product={}]{\inch}{\text{\textquotedbl}}
%\DeclareSIUnit[number-unit-product={\thinspace}]{\inch}{in}

\lstset{
  language=[LaTeX]{TeX},
  basicstyle=\ttfamily,
  keywordstyle=\fontfamily{lmss}\bfseries,
  tabsize=2,
  inputencoding=utf8/latin1,
  floatplacement=htb}

\lstdefinestyle{block}{
  basicstyle=\ttfamily,
  breaklines,
  breakindent=10pt}

\lstdefinestyle{numberedblock}{
  style=block,
  numbers=left,
  numberstyle=\scriptsize,
  xleftmargin=20pt,
  frame=leftline}


\newcommand{\instfig}[4][1.0]
{%
  \begin{minipage}{#2}
  \raggedleft\null\scalebox{#1}{\input{#4}}
  \end{minipage}%
  \hfill%
  \begin{minipage}{#3}
  \lstinputlisting[style=block]{#4}
  \end{minipage}%
}




\title{AviaTi\textit{k}Z}
\author{Mario Haustein}


\begin{document}

\maketitle
\tableofcontents

\section{Introduction}

This package provides a set of commands to draw graphics related for
theoretical flight training courses. It is base in the \texttt{tikz} package.
All drawing commands have to be placed inside a
\texttt{tikzpicture}~environment.

\section{Instruments}
Inside an aircraft a variety of instruments is used for navigation and
monitoring of the performance of a wide range of subsystems. The layout of
these instruments is internationally standardised. A quick and clear reading
of these instruments is essential for a safe and precise condution of flights.
High quality instrument drawings facilitate a successful lerning process.

The instruments are scaled in such a way to match the size of real instruments
(\SI[quotient-mode=fraction,fraction-function=\sfrac]{1/8}[3]{\inch},
\SI[quotient-mode=fraction,fraction-function=\sfrac]{1/4}[2]{\inch}, or
\SI[quotient-mode=fraction,fraction-function=\sfrac]{1/4}[1]{\inch} diameter
cutouts). Don't use PGFs coordinate transformations to rescale instruments as
it won't scale font size or line width for example. Instead you have to use
commands like \lstinline+\scalebox+ of the \texttt{graphicx} package.

\begin{figure}[!h]
\begin{subfigure}{\linewidth}
\centering
\instfig{0.55\linewidth}{0.4\linewidth}{documentation/figures/inst/sample_unscaled}
\caption{Unscaled Instrument}
\end{subfigure}

\bigskip

\begin{subfigure}{\linewidth}
\centering
\instfig{0.55\linewidth}{0.4\linewidth}{documentation/figures/inst/sample_scaled}
\caption{Scaled instrument}
\end{subfigure}

\caption{Drawing instruments}
\end{figure}


\subsection{Pitot-Static Instruments}

The pitot-static system consists of two pressure probes. The static port
measures the barometric pressure of the air around the aircraft. The pitot tube
measures the total pressure of the airflow in the direction of flight. The
difference between total and static pressure is called dynamic pressure.

\subsubsection{Altimeter}
Altimeters solely rely on the static pressure. Static pressure decreases the
higher the aircraft flies. The barometric pressure at ground is affected by the
current weather situation, so it isn't possible to relate pressure values
directly to altitude levels. To read the altitude above mean sea level the
altimeter has to be set up to the QNH reference value provided by ground
stations. The QNH is measured in hectopascal (\si{\hecto\Pa}).

The indicated altitude can be specified via the \texttt{altitude} keyword
argument (ref. figure~\ref{fig:inst:altimeter:basic}).

\begin{figure}[!h]
\centering
\instfig[0.65]{0.4\linewidth}{0.55\linewidth}{documentation/figures/inst/altimeter_basic}
\caption{Altimeter}
\label{fig:inst:altimeter:basic}
\end{figure}

The QNH can be set by the \texttt{qnh} keyword argument (ref.
figure~\ref{fig:inst:altimeter:qnh}). The QNH settings doesn't influence the
indicated altitude. If you want to demonstrate the effect of chaning the QNH
you have to perform the necessary calculations on your own.

\begin{figure}[!h]
\centering
\instfig[0.65]{0.4\linewidth}{0.55\linewidth}{documentation/figures/inst/altimeter_qnh}
\caption{QNH-setting of an altimeter}
\label{fig:inst:altimeter:qnh}
\end{figure}


\subsubsection{Variometer}

\begin{tikzpicture}
\aviainstvariometer[vspeed=-565]
\end{tikzpicture}

\subsubsection{Airspeed Indicator}

\begin{tikzpicture}
\aviainstasi[ias=102]
\end{tikzpicture}

\subsection{Gyroscopic Instruments}

\subsubsection{Heading Indicator}

\begin{tikzpicture}
\aviainsthi[hdg=125,hashdgbug=true,hdgbug=95]
\end{tikzpicture}

\begin{tikzpicture}
\aviainsthi[hdg=115]
\end{tikzpicture}

\subsubsection{Turn Coordinator}

\begin{tikzpicture}
\aviainsttc[turn=-3,slip=-1.5]
\end{tikzpicture}

\begin{tikzpicture}
\aviainsttc[]
\end{tikzpicture}


\subsubsection{Attitude Indicator}

\begin{tikzpicture}
\aviainstai[]
\end{tikzpicture}

\subsection{Radio Navigation}

\subsubsection{Automatic Direction Finder}

\begin{tikzpicture}
\aviainstadf[type=rbi,rb=90]
\end{tikzpicture}

\begin{tikzpicture}
\aviainstadf[type=mdi,brng hdg={30}{55}]
\end{tikzpicture}

\begin{tikzpicture}
\aviainstadf[type=mdi,hdg=45,rb=60]
\end{tikzpicture}

\subsubsection{Course Deviation Indicator}

\begin{tikzpicture}
\aviainstcdi[obs=30,ils/dev={0.2}{0.2}]
\end{tikzpicture}

\begin{tikzpicture}
\aviainstcdi[obs=215,vor/radial=30]
\end{tikzpicture}

\begin{tikzpicture}
\aviainstcdi[obs=215,type=vor,vor/radial=200]
\end{tikzpicture}

\subsection{Engine}

\subsubsection{Tachometer}

\subsubsection{Oil Pressure}

\subsubsection{Oil Temperature}

\subsubsection{Cylinder Head Temperature}

\subsubsection{Exhaust Gas Temperature}

\subsubsection{Carburetor Temperature}

\subsubsection{Fuel Quantity}

\subsubsection{Fuel Flow}

\subsubsection{Manifold Pressure}


\end{document}

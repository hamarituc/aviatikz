\documentclass[11pt,a4paper]{article}


\usepackage[T1]{fontenc}
\usepackage[utf8]{inputenc}
\usepackage{newpxtext}
\usepackage{newpxmath}
\usepackage{textcomp}
\usepackage{parskip}
\usepackage{a4wide}
\usepackage{float}
\usepackage{subcaption}
\usepackage{tikz}
\usepackage{aviatikz}
\usepackage{xfrac}
\usepackage{amsmath}
\usepackage{siunitx}
\usepackage{listings}
\usepackage{listingsutf8}
\usepackage{booktabs}
\usepackage{tabularx}

\floatplacement{figure}{htb}
\floatplacement{table}{htb}


\DeclareSIUnit[number-unit-product={}]{\inch}{\text{\textquotedbl}}
%\DeclareSIUnit[number-unit-product={\thinspace}]{\inch}{in}
\DeclareSIUnit{\feet}{ft}
\DeclareSIUnit{\KIAS}{KIAS}

\sisetup{per-mode = symbol}


\lstset{
  language=[LaTeX]{TeX},
  basicstyle=\ttfamily,
  keywordstyle=\fontfamily{lmss}\bfseries,
  tabsize=2,
  inputencoding=utf8/latin1,
  floatplacement=htb}

\lstdefinestyle{block}{
  basicstyle=\ttfamily,
  breaklines,
  breakindent=10pt}

\lstdefinestyle{numberedblock}{
  style=block,
  numbers=left,
  numberstyle=\scriptsize,
  xleftmargin=20pt,
  frame=leftline}


\newcommand{\instfig}[4][1.0]
{%
  \begin{minipage}{#2}
  \raggedleft\null\scalebox{#1}{\input{#4}}
  \end{minipage}%
  \hfill%
  \begin{minipage}{#3}
  \lstinputlisting[style=block]{#4}
  \end{minipage}%
}



\newcommand{\lara}[1]{\textnormal{\textlangle#1\textrangle}}


\title{AviaTi\textit{k}Z}
\author{Mario Haustein}


\begin{document}

\maketitle
\tableofcontents

\section{Introduction}

This package provides a set of commands to draw graphics related for
theoretical flight training courses. It is base in the \texttt{tikz} package.
All drawing commands have to be placed inside a
\texttt{tikzpicture}~environment.

\section{Instruments}
Inside an aircraft a variety of instruments is used for navigation and
monitoring of the performance of a wide range of subsystems. The layout of
these instruments is internationally standardised. A quick and clear reading
of these instruments is essential for a safe and precise condution of flights.
High quality instrument drawings facilitate a successful lerning process.

The instruments are scaled in such a way to match the size of real instruments
(\SI[quotient-mode=fraction,fraction-function=\sfrac]{1/8}[3]{\inch},
\SI[quotient-mode=fraction,fraction-function=\sfrac]{1/4}[2]{\inch}, or
\SI[quotient-mode=fraction,fraction-function=\sfrac]{1/4}[1]{\inch} diameter
cutouts). Don't use PGFs coordinate transformations to rescale instruments as
it won't scale font size or line width for example. Instead you have to use
commands like \lstinline+\scalebox+ of the \texttt{graphicx} package.

\begin{figure}[!h]
\begin{subfigure}{\linewidth}
\centering
\instfig{0.55\linewidth}{0.4\linewidth}{documentation/figures/inst/sample_unscaled}
\caption{Unscaled Instrument}
\end{subfigure}

\bigskip

\begin{subfigure}{\linewidth}
\centering
\instfig{0.55\linewidth}{0.4\linewidth}{documentation/figures/inst/sample_scaled}
\caption{Scaled instrument}
\end{subfigure}

\caption{Drawing instruments}
\end{figure}


\clearpage
\subsection{Pitot-Static Instruments}

The pitot-static system consists of two pressure probes. The static port
measures the barometric pressure of the air around the aircraft. The pitot tube
measures the total pressure of the airflow in the direction of flight. The
difference between total and static pressure is called dynamic pressure.

\subsubsection{Altimeter}
Altimeters solely rely on the static pressure. Static pressure decreases the
higher the aircraft flies. The barometric pressure at ground is affected by the
current weather situation, so it isn't possible to relate pressure values
directly to altitude levels. To read the altitude above mean sea level the
altimeter has to be set up to the QNH reference value provided by ground
stations. The QNH is measured in hectopascal (\si{\hecto\Pa}).

The indicated altitude can be specified via the \texttt{altitude} keyword
argument (ref. figure~\ref{fig:inst:altimeter:basic}).

\begin{figure}[!h]
\centering
\instfig[0.65]{0.4\linewidth}{0.55\linewidth}{documentation/figures/inst/altimeter_basic}
\caption{Altimeter}
\label{fig:inst:altimeter:basic}
\end{figure}

The QNH can be set by the \texttt{qnh} keyword argument (ref.
figure~\ref{fig:inst:altimeter:qnh}). The QNH settings doesn't influence the
indicated altitude. If you want to demonstrate the effect of chaning the QNH
you have to perform the necessary calculations on your own.

\begin{figure}[!h]
\centering
\instfig[0.65]{0.4\linewidth}{0.55\linewidth}{documentation/figures/inst/altimeter_qnh}
\caption{QNH-setting of an altimeter}
\label{fig:inst:altimeter:qnh}
\end{figure}


\subsubsection{Variometer}
The vertical speed indicator (or sometimes called variometer) shows the climb
and descend rates of an aircraft by measuring the derivative of the static
pressure. You can specify the vertical speed by the \texttt{vspeed} keyword
argument (ref. figure~\ref{fig:inst:vsi:basic}). Vertical speeds are specified
in \si{\feet\per\minute}. Since the variometer is a quite simple instrument
there are no further parameters.

\begin{figure}[!h]
\centering
\instfig[0.7]{0.4\linewidth}{0.55\linewidth}{documentation/figures/inst/vsi_basic}
\caption{Vertical Speed Indicator}
\label{fig:inst:vsi:basic}
\end{figure}


\subsubsection{Airspeed Indicator}
The total pressure is affected by the airspeed of the aircraft while the static
pressure is independent of the airspeed. This allows to measure the airspeed by
comparing total to static pressure. Values shown on an airspeed indicator have
to be corrected for several inflencing factors. These correntions lie beyond
the scope of this commands. All speed values are considered as indicated values
(IAS).

The indicated airspeed is specified be the \texttt{ias} keyword argument.
Figure~\ref{fig:inst:asi:c172n} shows \SI{102}{\KIAS}.

\begin{figure}[!h]
\centering
\instfig[0.7]{0.4\linewidth}{0.55\linewidth}{documentation/figures/inst/asi_c172n}
\caption{Air Speed Indicator}
\label{fig:inst:asi:c172n}
\end{figure}

The airspeed indicator is an important performance instrument of the aircraft.
Thus it carries several markings specific to the type of aircraft. These
markings highlight different operating ranges depending on characteristic speed
values shown in table~\ref{tab:inst:asi:markings}.

\begin{table}
\centering
\begin{tabular}{ccl}
\toprule
Marking    & V-Speed                             & Remarks                                       \\
\midrule
white arc  & $V_\mathrm{S_0}$ -- $V_\mathrm{FE}$ & Flap operating range                          \\
green arc  & $V_\mathrm{S_1}$ -- $V_\mathrm{NO}$ & Normal operating range                        \\
yellow arc & $V_\mathrm{NO}$  -- $V_\mathrm{NE}$ & Operation with caution and only in smooth air \\
red mark   & $V_\mathrm{NE}$                     & Maximum speed for all operations              \\
\bottomrule
\end{tabular}
\caption{Airspeed indicator markings}
\label{tab:inst:asi:markings}
\end{table}

The airspeed indicator can be parameterized by a variety of arguments. Default
values are set up for a Cessna~172N aircraft. The configurable speed parameters
are shown in table~\ref{tab:inst:asi:speeds}.

\begin{table}
\centering
\begin{tabular}{ccl}
\toprule
Keyword             & V-Speed          & Remarks                              \\
\midrule
\texttt{speeds/vs0} & $V_\mathrm{S_0}$ & Stall speed in landing configuration \\
\texttt{speeds/vs1} & $V_\mathrm{S_1}$ & Stall speed in any configuration     \\
\texttt{speeds/vfe} & $V_\mathrm{FE}$  & Maximum speed with flaps extended    \\
\texttt{speeds/vno} & $V_\mathrm{NO}$  & Maximum structural cruising speed    \\
\texttt{speeds/vne} & $V_\mathrm{NE}$  & Never exceed speed                   \\
\bottomrule
\end{tabular}
\caption{Airspeed indicator V-speeds}
\label{tab:inst:asi:speeds}
\end{table}

Figure~\ref{fig:inst:asi:c182t} shows the airspeed indicator of Cessna~182T.

\begin{figure}[!h]
\centering
\instfig[0.7]{0.4\linewidth}{0.55\linewidth}{documentation/figures/inst/asi_c182t}
\caption{Air Speed Indicator for a Cessna~182T}
\label{fig:inst:asi:c182t}
\end{figure}

To change the indicating range of the instrument you can use the parameters
shown in table~\ref{tab:inst:asi:range}. Figure~\ref{fig:inst:asi:range} shows
the same airspeed indicator configuration as in figure~\ref{fig:inst:asi:c172n}
but with a changed face layout.

\begin{table}
\centering
\begin{tabular}{cl}
\toprule
Keyword            & Remarks                                                   \\
\midrule
\texttt{face/vmin} & Lowest speed value                                        \\
\texttt{face/vmax} & Highes spped value                                        \\
\texttt{face/amin} & Position angle of lower bound measured clockwise from top \\
\texttt{face/amax} & Position angle of upper bound measured clockwise from top \\
\bottomrule
\end{tabular}
\caption{Airspeed indicator V-speeds}
\label{tab:inst:asi:range}
\end{table}

\begin{figure}[!h]
\centering
\instfig[0.7]{0.4\linewidth}{0.55\linewidth}{documentation/figures/inst/asi_range}
\caption{Air Speed Indicator with a changed face}
\label{fig:inst:asi:range}
\end{figure}


\clearpage
\subsection{Gyroscopic Instruments}

Gyroscopic instruments using spinning masses driven either pneumatically by a
suction pump or electrically. They base on the effect of precession to display
they orientation or turn rate of the aircraft uninfluenced by acceleration
forces.

\subsubsection{Heading Indicator}
The heading indicator show the direction into which the longitudinal axis of
the aircraft points. It is usually manually synchronized to the reading of a
magnetic compass.

The heading is specified via the \texttt{hdg} keyword argument. Some heading
indicators integrate an adjustable a heading bug. This heading bug acts as a
source for an autopilot. The heading bug must be activated via the
\texttt{hashdgbug=true} option and can be adjusted by the \texttt{hdgbug}
keyword argument. Figure~\ref{fig:inst:hi:base} shows both types of heading
indicators.

\begin{figure}[!h]
\begin{subfigure}{\linewidth}
\centering
\instfig[0.7]{0.4\linewidth}{0.55\linewidth}{documentation/figures/inst/hi_nobug}
\caption{without heading bug}
\label{fig:inst:hi:base_nobug}
\end{subfigure}

\medskip

\begin{subfigure}{\linewidth}
\centering
\instfig[0.7]{0.4\linewidth}{0.55\linewidth}{documentation/figures/inst/hi_bug}
\caption{with heading bug}
\label{fig:inst:hi:base_bug}
\end{subfigure}

\caption{Heading indicator}
\label{fig:inst:hi:base}
\end{figure}


\subsubsection{Turn Coordinator}
The turn coordinator shows the turning rate around the yaw axis. Standard rate
turns are performed with \SI{3}{\degree\per\second}. A full turn requires
\SI{2}{\minute}. An aircraft turns with a standard rate if the aircraft symbol
aligns with one of the lower bold marking lines. The small dots mark half and
double rate turns.

Turn rates can be specified in two different ways.

\begin{itemize}
\item \texttt{turn=\lara{rate}} specifies the turn rate in degrees per second.
\item \texttt{stdrate=\lara{rate}} specifies the turn rate in multiples of standard rate turns.
\end{itemize}

Use positive values for right turns. The inclinometer is adjusted by the
\texttt{slip} keyword argument. A positive value deflects the ball to the left.
A value of one deflects the ball just as much as necessary to leave the center
marker. Figure~\ref{fig:inst:tc:base} shows two examples.

\begin{figure}[!h]
\begin{subfigure}{\linewidth}
\centering
\instfig[0.7]{0.4\linewidth}{0.55\linewidth}{documentation/figures/inst/tc_straight_coord}
\caption{straight and coordinated flight}
\end{subfigure}

\medskip

\begin{subfigure}{\linewidth}
\centering
\instfig[0.7]{0.4\linewidth}{0.55\linewidth}{documentation/figures/inst/tc_stdleft_skid}
\caption{skidding standard left turn}
\end{subfigure}

\caption{Turn coordinator}
\label{fig:inst:tc:base}
\end{figure}

The warning flag of the turn coordinator is triggered on malfunctions to advise
the pilot not to rely on the instrument readings. To trigger the warning set
the \texttt{flag} keyword.

\begin{figure}
\centering
\instfig[0.7]{0.4\linewidth}{0.55\linewidth}{documentation/figures/inst/tc_fail}
\caption{Turn coordinator with triggered warning flag}
\label{fig:inst:tc:flag}
\end{figure}


\subsubsection{Attitude Indicator}

\begin{tikzpicture}
\aviainstai[]
\end{tikzpicture}

\clearpage
\subsection{Radio Navigation}

\subsubsection{Automatic Direction Finder}

\begin{tikzpicture}
\aviainstadf[type=rbi,rb=90]
\end{tikzpicture}

\begin{tikzpicture}
\aviainstadf[type=mdi,brng hdg={30}{55}]
\end{tikzpicture}

\begin{tikzpicture}
\aviainstadf[type=mdi,hdg=45,rb=60]
\end{tikzpicture}

\subsubsection{Course Deviation Indicator}

\begin{tikzpicture}
\aviainstcdi[obs=30,ils/dev={0.2}{0.2}]
\end{tikzpicture}

\begin{tikzpicture}
\aviainstcdi[obs=215,vor/radial=30]
\end{tikzpicture}

\begin{tikzpicture}
\aviainstcdi[obs=215,type=vor,vor/radial=200]
\end{tikzpicture}

\subsection{Engine}

\subsubsection{Tachometer}

\subsubsection{Oil Pressure}

\subsubsection{Oil Temperature}

\subsubsection{Cylinder Head Temperature}

\subsubsection{Exhaust Gas Temperature}

\subsubsection{Carburetor Temperature}

\subsubsection{Fuel Quantity}

\subsubsection{Fuel Flow}

\subsubsection{Manifold Pressure}


\end{document}

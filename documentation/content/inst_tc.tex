The turn coordinator shows the turning rate around the yaw axis. Standard rate
turns are performed with \SI{3}{\degree\per\second}. A full turn requires
\SI{2}{\minute}. An aircraft turns with a standard rate if the aircraft symbol
aligns with one of the lower bold marking lines. The small dots mark half and
double rate turns.

Turn rates can be specified in two different ways.

\begin{itemize}
\item \texttt{turn=\lara{rate}} specifies the turn rate in degrees per second.
\item \texttt{stdrate=\lara{rate}} specifies the turn rate in multiples of standard rate turns.
\end{itemize}

Use positive values for right turns. The inclinometer is adjusted by the
\texttt{slip} keyword argument. A positive value deflects the ball to the left.
A value of one deflects the ball just as much as necessary to leave the center
marker. Figure~\ref{fig:inst:tc:base} shows two examples.

\begin{figure}[!h]
\begin{subfigure}{\linewidth}
\centering
\instfig[0.7]{0.4\linewidth}{0.55\linewidth}{documentation/figures/inst/tc_straight_coord}
\caption{straight and coordinated flight}
\end{subfigure}

\medskip

\begin{subfigure}{\linewidth}
\centering
\instfig[0.7]{0.4\linewidth}{0.55\linewidth}{documentation/figures/inst/tc_stdleft_skid}
\caption{skidding standard left turn}
\end{subfigure}

\caption{Turn coordinator}
\label{fig:inst:tc:base}
\end{figure}

The warning flag of the turn coordinator is triggered on malfunctions to advise
the pilot not to rely on the instrument readings. To trigger the warning set
the \texttt{flag} keyword.

\begin{figure}
\centering
\instfig[0.7]{0.4\linewidth}{0.55\linewidth}{documentation/figures/inst/tc_fail}
\caption{Turn coordinator with triggered warning flag}
\label{fig:inst:tc:flag}
\end{figure}

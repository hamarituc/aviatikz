Altimeters solely rely on the static pressure. Static pressure decreases the
higher the aircraft flies. The barometric pressure at ground is affected by the
current weather situation, so it isn't possible to relate pressure values
directly to altitude levels. To read the altitude above mean sea level the
altimeter has to be set up to the QNH reference value provided by ground
stations. The QNH is measured in hectopascal (\si{\hecto\Pa}).

The indicated altitude can be specified via the \texttt{altitude} keyword
argument (ref. figure~\ref{fig:inst:altimeter:basic}).

\begin{figure}[!h]
\centering
\instfig[0.65]{0.4\linewidth}{0.55\linewidth}{documentation/figures/inst/altimeter_basic}
\caption{Altimeter}
\label{fig:inst:altimeter:basic}
\end{figure}

The QNH can be set by the \texttt{qnh} keyword argument (ref.
figure~\ref{fig:inst:altimeter:qnh}). The QNH settings doesn't influence the
indicated altitude. If you want to demonstrate the effect of chaning the QNH
you have to perform the necessary calculations on your own.

\begin{figure}[!h]
\centering
\instfig[0.65]{0.4\linewidth}{0.55\linewidth}{documentation/figures/inst/altimeter_qnh}
\caption{QNH-setting of an altimeter}
\label{fig:inst:altimeter:qnh}
\end{figure}

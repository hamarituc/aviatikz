The heading indicator show the direction into which the longitudinal axis of
the aircraft points. It is usually manually synchronized to the reading of a
magnetic compass.

The heading is specified via the \texttt{hdg} keyword argument. Some heading
indicators integrate an adjustable a heading bug. This heading bug acts as a
source for an autopilot. The heading bug must be activated via the
\texttt{hashdgbug=true} option and can be adjusted by the \texttt{hdgbug}
keyword argument. Figure~\ref{fig:inst:hi:base} shows both types of heading
indicators.

\begin{figure}[!h]
\begin{subfigure}{\linewidth}
\centering
\instfig[0.7]{0.4\linewidth}{0.55\linewidth}{documentation/figures/inst/hi_nobug}
\caption{without heading bug}
\label{fig:inst:hi:base_nobug}
\end{subfigure}

\medskip

\begin{subfigure}{\linewidth}
\centering
\instfig[0.7]{0.4\linewidth}{0.55\linewidth}{documentation/figures/inst/hi_bug}
\caption{with heading bug}
\label{fig:inst:hi:base_bug}
\end{subfigure}

\caption{Heading indicator}
\label{fig:inst:hi:base}
\end{figure}

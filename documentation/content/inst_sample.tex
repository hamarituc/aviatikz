Inside an aircraft a variety of instruments is used for navigation and
monitoring of the performance of a wide range of subsystems. The layout of
these instruments is internationally standardised. A quick and clear reading
of these instruments is essential for a safe and precise condution of flights.
High quality instrument drawings facilitate a successful lerning process.

The instruments are scaled in such a way to match the size of real instruments
(\SI[quotient-mode=fraction,fraction-function=\sfrac]{1/8}[3]{\inch},
\SI[quotient-mode=fraction,fraction-function=\sfrac]{1/4}[2]{\inch}, or
\SI[quotient-mode=fraction,fraction-function=\sfrac]{1/4}[1]{\inch} diameter
cutouts). Don't use PGFs coordinate transformations to rescale instruments as
it won't scale font size or line width for example. Instead you have to use
commands like \lstinline+\scalebox+ of the \texttt{graphicx} package.

\begin{figure}[!h]
\begin{subfigure}{\linewidth}
\centering
\instfig{0.55\linewidth}{0.4\linewidth}{documentation/figures/inst/sample_unscaled}
\caption{Unscaled Instrument}
\end{subfigure}

\bigskip

\begin{subfigure}{\linewidth}
\centering
\instfig{0.55\linewidth}{0.4\linewidth}{documentation/figures/inst/sample_scaled}
\caption{Scaled instrument}
\end{subfigure}

\caption{Drawing instruments}
\end{figure}
